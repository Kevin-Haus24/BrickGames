\documentclass{article}
\usepackage[utf8]{inputenc}
\usepackage[T2A]{fontenc}
\usepackage[russian,english]{babel}
\usepackage{geometry}
\geometry{a4paper, margin=2.5cm}
\usepackage{hyperref}
\title{Tetris Game Documentation}
\author{BrickGame v1.0}
\date{\today}
\begin{document}
\maketitle

\section{Overview}
Tetris (BrickGame v1.0) — консольная реализация классической игры Тетрис на языке C11 с использованием ncurses. Проект построен по принципам структурного программирования и разделён на библиотеку логики и терминальный интерфейс.

\section{Requirements}
\subsection{System Requirements}
\begin{itemize}
  \item \textbf{OS:} Linux/Unix-подобные системы
  \item \textbf{Компилятор:} GCC с поддержкой стандарта C11
  \item \textbf{Терминал:} с поддержкой Unicode (UTF-8)
\end{itemize}

\subsection{Required Libraries}
\begin{itemize}
  \item \textbf{build-essential} — набор основных инструментов разработки (gcc, make, libc6-dev)
  \item \textbf{libncursesw5-dev} — библиотека для работы с терминальным интерфейсом (широкие символы)
  \item \textbf{check} — фреймворк для unit-тестирования на C
\end{itemize}

\subsection{Development Tools}
\begin{itemize}
  \item \textbf{valgrind} — анализ утечек памяти и профилирование
  \item \textbf{lcov} — генерация отчётов о покрытии кода
  \item \textbf{clang-format} — форматирование кода по стандартам (версия 18.1.8)
  \item \textbf{cppcheck} — статический анализ кода (версия 2.13.0)
\end{itemize}

\subsection{Documentation Tools}
\begin{itemize}
  \item \textbf{texlive-latex-extra} — набор LaTeX
  \item \textbf{texlive-lang-cyrillic} — поддержка кириллицы в LaTeX
\end{itemize}

\subsection{Installation Commands}
\begin{verbatim}
# Ubuntu/Debian (полная установка):
sudo apt update && sudo apt install -y \
    build-essential \
    libncursesw5-dev \
    check \
    valgrind \
    lcov \
    texlive-latex-extra \
    texlive-lang-cyrillic

# Установка LLVM/Clang 18 и clang-format:
wget https://apt.llvm.org/llvm.sh && \
chmod +x llvm.sh && \
sudo ./llvm.sh 18 && \
rm llvm.sh

# Установка clang-format-18 и создание символической ссылки:
sudo apt install clang-format-18 && \
sudo ln -sf /usr/bin/clang-format-18 /usr/bin/clang-format

# Для cppcheck (сборка из исходников):
git clone https://github.com/danmar/cppcheck.git && \
cd cppcheck && \
git checkout tags/2.13.0 -b v2.13.0 && \
mkdir build && cd build && \
sudo apt install -y cmake && \
cmake .. -DCMAKE_INSTALL_PREFIX=/usr/local && \
make -j$(nproc) && \
sudo make install
\end{verbatim}

\section{Features}
\begin{itemize}
  \item 7 типов фигур (I, O, T, S, Z, J, L)
  \item Вращение, перемещение, ускоренное падение фигур
  \item Очистка заполненных линий
  \item Показ следующей фигуры
  \item Подсчёт очков и уровней, сохранение рекорда
  \item Завершение игры при заполнении верхней границы
  \item Управление с клавиатуры (8 кнопок)
  \item Размер поля: 10×20
\end{itemize}

\section{Build and Run}
\begin{verbatim}
make all         # Сборка игры
make run         # Запуск игры
make test        # Запуск автотестов
make install     # Установка в ./usr/local/bin/
make uninstall   # Удаление установленной игры
make clean       # Очистка сборочных файлов
make doc         # Генерация документации
make dvi         # Генерация DVI-документации
make pdf         # Генерация PDF-документации
make html        # Генерация HTML-документации
make dist        # Создание дистрибутива tar.gz
make gcov_report # Генерация отчёта о покрытии кода

make check       # Полная проверка кода (форматирование, анализ, память)
make clang       # Проверка форматирования кода
make cppcheck    # Статический анализ кода
make mem         # Проверка утечек памяти
\end{verbatim}

\section{Controls}
\begin{itemize}
  \item \textbf{S} — старт игры
  \item \textbf{P} — пауза/продолжить
  \item \textbf{Q} — выход
  \item \textbf{Стрелки влево/вправо} — движение фигуры
  \item \textbf{Стрелка вниз} — ускоренное падение
  \item \textbf{Пробел} — вращение фигуры
\end{itemize}

\section{Game Mechanics}
\begin{itemize}
  \item \textbf{Очки:} 1 линия — 100, 2 — 300, 3 — 700, 4 — 1500
  \item \textbf{Уровень:} +1 за каждые 600 очков (максимум 10)
  \item \textbf{Скорость:} увеличивается с ростом уровня
  \item \textbf{Рекорд:} сохраняется между сессиями
\end{itemize}

\section{Project Structure}
\begin{verbatim}
brick_game/tetris/    # Логика игры (библиотека)
gui/cli/              # Терминальный интерфейс
tests/                # Автотесты
doc/                  # Документация
\end{verbatim}

\section{API Reference}
\begin{itemize}
  \item \texttt{void userInput(UserAction\_{}t action, bool hold);} — обработка ввода пользователя
  \item \texttt{GameInfo\_{}t updateCurrentState();} — получить текущее состояние игры
  \item \texttt{void initGame();} — инициализация новой игры
  \item \texttt{void freeGame();} — освобождение ресурсов
\end{itemize}

\end{document}